%%%%%%%%%%%%%%%%%%%%%
%% Tipos de Usuários %%
%%%%%%%%%%%%%%%%%%%%%

\section{Introdução}
O SAEC possui 4 diferentes tipos de usuários, cada qual com suas funções e permissões.

%%%%%%%%%%%%%%%%%%%%%

\section{God}
No SAEC não existe uma categoria diferente de usuário para o usuário \textbf{criador}, e ele é visto como um \textit{administrator} perante o sistema. No entando, consideramos que durante o processo de configuração do SAEC, o primeiro \textit{admin}, que executa o processo, é o criador.

Este conceito é importante, pois o usuário \textit{administrator} não pode ter acesso às páginas da etapa de configuração do sistema após encerrada essa etapa.

%%%%%%%%%%%%%%%%%%%%%

\section{Administrator}
TODO

%%%%%%%%%%%%%%%%%%%%%

\section{Operator}
O usuário \textbf{operador} está fortemente relacionado à uma instituição. Ele é responsável por garantir que a relação entre o SAEC e a instituição que representa esteja correta.

%%%%%%%%%%%%%%%%%%%%%

\section{Default}
O usuário \textbf{default}, ou usuário comum, é aquele que não está \textit{logado}. Trata-se dos clientes do serviço, que o acessam para a emissão de certificados.