%%%%%%%%%%%%%%%%%%%%%
%% Instalação %%
%%%%%%%%%%%%%%%%%%%%%

\section{Introdução}
Esta seção tem como objetivo mostrar, passo a passo, como é feita a instalação e configuração do ambiente de desenvolvimento do SAEC. Deve-se levar em consideração que este documento foi produzido com base na instalação da versão 2.3 do SAEC, em um Ubuntu 14.04 de 64 bits.

%%%%%%%%%%%%%%%%%%%%%

\section{Repositório}
O código da aplicação encontra-se no GitLab do LabSEC e para acessá-lo é necessário possuir cadastro. É preciso também criar uma chave SSH e adicioná-la ao git, para que seja possível fazer o download do código da aplicação e realizar commits.

\subsection{Cadastro no GitLab LabSEC}
TODO

\subsection{Chave SSH}
O link abaixo deve ser utilizado para adicionar chaves SSH à sua conta do GitLab e verificar as que já foram adicionadas.

\begin{lstlisting}
    https://gitlab.labsec.ufsc.br/profile/keys
\end{lstlisting}

Para criar uma chave, o comando a seguir deve ser utilizado, com o seu email utilizado no GitLab.

\begin{lstlisting}[language=bash]
    $ ssh-keygen -t rsa -C "your.email@example.com" -b 4096
\end{lstlisting}

Recomenda-se utilizar o diretório padrão para salvar a chave (pressionando \textit{enter} quando o caminho for solicitado), e uma senha de sua escolha. Com a chave gerada, deve-se adicioná-la ao GitLab. Para imprimir a chave no terminal e copiá-la com \textit{Ctrl+Shift+C}, utilize o comando:

\begin{lstlisting}[language=bash]
    $ cat ~/.ssh/id_rsa.pub
\end{lstlisting}

\subsection{Download do Código}
Para baixar o código da aplicação e contribuir para o desenvolvimento da aplicação, é necessário fazer o download do \textit{git}.

\begin{lstlisting}[language=bash]
	$ sudo apt-get install git
\end{lstlisting}

No diretório de sua escolha, clone o repositório. Será criada uma pasta \textit{code}, que pode ser renomeada, com o código presente na branch \textit{master} do repositório.

\begin{lstlisting}[language=bash]
	$ git clone git@gitlab.labsec.ufsc.br:saec/code.git
\end{lstlisting}

Para fazer checkout do branch \textit{dev}, os seguintes comandos devem ser utilizados dentro da pasta onde o código está presente:

\begin{lstlisting}[language=bash]
	$ git fetch
	$ git checkout -b dev origin/dev
\end{lstlisting}

%%%%%%%%%%%%%%%%%%%%%

\section{Instalando Dependências}
Algumas dependências precisam ser instaladas para que seja possível utilizar o SAEC.

\begin{lstlisting}[language=bash]
	$ sudo apt-get install postgresql
	$ sudo apt-get install php5 php5-sqlite php5-ldap
	$ sudo apt-get install php5-mcrypt php5-dev php5-pgsql
	$ sudo apt-get install libapache2-mod-shib2
	$ sudo apt-get install libapache2-modsecurity
	$ sudo apt-get install libp11-dev
\end{lstlisting}

Deve-se, também, habilitar módulos do apache e do php.

\begin{lstlisting}
        $ sudo a2enmod ssl
        $ sudo a2enmod shib2
        $ sudo a2enmod rewrite
        $ sudo a2enmod headers
        $ sudo php5enmod mcrypt
\end{lstlisting}

TODO: ibcryptosec!

%%%%%%%%%%%%%%%%%%%%%

\section{Gerando Chave do Servidor HTTPS}

TODO

%%%%%%%%%%%%%%%%%%%%%

\section{Configurando o Banco de Dados}

TODO

%%%%%%%%%%%%%%%%%%%%%

\section{Apache Virtual Host}

TODO

%%%%%%%%%%%%%%%%%%%%%

\section{Testando Instalação}

TODO